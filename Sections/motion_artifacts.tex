\documentclass[class=article, crop=false]{standalone}

\begin{document}

The type of motion artifact and its severity depends on the magnitude and type of motion, as well as its temporal relation to the data acquisition method. In the following discussion, motion induced blurring and ghosting will be covered first, with other signal deterioration effects such as spin de-phasing covered in the second subsection. Blurring and ghosting artifacts due to motion are generally caused by spatial encoding inconsistencies, while loss of signal contrast and aberrations in signal intensity are related to how motion affects signal generation before and during acquisition.

\subsection{Blurring and Ghosting}

Perhaps the most obvious sign that motion has occurred during a scan is the presence of blurring and ghosts in the reconstructed image. Blurring is the smearing of edges, while ghosting is the presence of repeating image structures, usually in the phase encoding direction. The presence of these artifacts can be understood by reconstructing motion affected k-space data under the assumption that the spin density has not moved.

\subsubsection*{Homogeneous Transformations}
Following the approach of Zahneisen and Ernst \parencite*{Zahneisen2016}, a homogeneous coordinate formalism will be used to describe motion-affected k-space. Homogeneous coordinates are widely used in computer vision, and allow for the concise representation of projective, affine and rigid body transformations. A point in three dimensional space $\textbf{r} = \begin{bmatrix} x,&y,&z\end{bmatrix}^{T}$ is represented in homogeneous coordinates as the augmented vector $\tilde{\textbf{r}} = \begin{bmatrix}x,&y,&z,&1\end{bmatrix}^{T}$. This augmented vector can then be operated on by a $4\times4$ homogeneous matrix \textbf{A}.
\begin{equation}
	\label{eq1}
	\tilde{\textbf{r}}^{'} = \textbf{A}\tilde{\textbf{r}}
\end{equation}
For our purposes, \textbf{A} represents either a rigid body or affine (shearing, scaling, as well as rigid body) transformation. Rigid body, or bulk motion is characterised by six degrees of freedom (three for translation, and three for rotation). The homogeneous matrix representation of this type of motion can be written as $\textbf{A} = \begin{bmatrix} \textbf{R} & \textbf{d} \\ 0 & 1 \end{bmatrix}$, where \textbf{R} is $3\times3$ rotation matrix and \textbf{d} is a $3\times1$ translation vector. It can thus be verified that Eq. (\ref{eq1}) is the homogeneous representation of $\textbf{r}^{'} = \textbf{R} \textbf{r} + \textbf{d}$.
\par
Affinities are a higher order group of transformations than rigid body transforms, and require twelve degrees of freedom to characterise. They can also be represented as a $4\times4$ matrix \textbf{A}, but with the $3\times3$ matrix \textbf{R} from the rigid body case replaced with a matrix \textbf{S}, which has the same dimensions. \textbf{S} contains not only the parameters that describe rotation, but also shearing and scaling.
\par
\textit{In vivo} motion can be quite complex, and in the case of respiratory or cardiac motion, time-dependent deformation fields would be needed for a full characterisation. In practice however, this complete description is often not necessary, and an affine model (with other restrictions) is sufficient. Rigid body transformations are used to model head motion in neural imaging \parencite{Godenschweger2016}, while both rigid body and affine models have been used to describe motion in cardiac and abdominal imaging \parencite{Nehrke2005,Manke2002,Pipe1999}. While affine models for \textit{in vivo} motion are quite useful, they do not take into account other sources of motion such as pulsed flow. In abdominal and chest imaging, it is also important to keep in mind that the affine model is only useful as a local approximation for a region of interest, and generally not applicable to the entire field of view.
\par
Nevertheless, in the following investigation we will assume that the points or voxels $\tilde{\textbf{r}}$ in the imaging volume are undergoing motion that can be modelled as an affine transformation. The movement of $\tilde{\textbf{r}}$ to a new position $\tilde{\textbf{r}}^{'}$ can therefore be described by Eq. (\ref{eq1}).

\subsubsection*{Motion and K-space}

\subsection{Signal and Contrast Corruption}


\end{document}