\documentclass[class=article, crop=false]{standalone}

\begin{document}

The type of motion artifact and its severity depends on the magnitude and type of motion, as well as its temporal relation to the pulse sequence. This makes a general description of motion effects quite difficult, as the same type of motion may have different effects across scans. In the following discussion, a broad distinction will be made between \textit{inter-view} and \textit{intra-view} effects. Inter-view effects are the result of movement during the latent periods between data acquisition and the following excitation. For motion that occurs during data acquisition (when spatial encoding gradients are switched on), intra-view effects will be present.

\subsection{Inter-view Effects}

For many gradient echo or spin echo sequences, the interval over which the gradients are switched on is much smaller than the repetition time, as well as the characteristic timescales of \textit{in vivo} motion. In this case, it can be assumed that no motion occurs during gradient encoding, and that the spin density is essentially 'displaced' from one encoding period to the next. Here, we will be mainly discussing bulk motion, which is the movement of large structures within the image, or the entire field of view itself. 
\par
Perhaps the most obvious sign that motion has occurred during a scan is the presence of blurring and ghosts in the reconstructed image. Blurring is the smearing of edges, while ghosting is the presence of repeated structures. Both of these artifacts can arise due to inconsistencies in k-space caused by bulk motion. Following the approach of Zahneisen and Ernst \parencite*{Zahneisen2016}, a homogeneous coordinate formalism will be used to analyse acquired k-space when motion is present.

\subsubsection*{Homogeneous Representation of Bulk Motion}
Homogeneous coordinates are widely used in computer vision, and allow for the concise representation of projective, affine and rigid body transformations. A point in three dimensional space $\textbf{r} = \begin{bmatrix} x,&y,&z\end{bmatrix}^{T}$ is represented in homogeneous coordinates by the augmented vector $\tilde{\textbf{r}} = \begin{bmatrix}x,&y,&z,&1\end{bmatrix}^{T}$. This augmented vector can then be operated on by a $4\times4$ homogeneous matrix \textbf{A}.
\begin{equation}
	\label{eq1}
	\tilde{\textbf{r}}^{'} = \textbf{A}\tilde{\textbf{r}}
\end{equation}
For our purposes, \textbf{A} represents either a rigid body (rotations and translations only) or affine (shearing, scaling, as well as rigid body) transformation. Rigid body motion is characterised by six degrees of freedom (three for translation, and three for rotation). The homogeneous matrix representation of this type of motion can be written as $\textbf{A} = \begin{bmatrix} \textbf{R} & \textbf{d} \\ \textbf{0} & 1 \end{bmatrix}$, where \textbf{R} is $3\times3$ rotation matrix and \textbf{d} is a $3\times1$ translation vector. It can thus be verified that Eq. (\ref{eq1}) is the homogeneous representation of $\textbf{r}^{'} = \textbf{R} \textbf{r} + \textbf{d}$.
\par
Affinities are a higher order group of transformations than rigid body transforms, and require twelve degrees of freedom to characterise. They can also be represented as a $4\times4$ matrix \textbf{A}, but with the $3\times3$ matrix \textbf{R} from the rigid body case replaced with a matrix \textbf{S}, which has the same dimensions. \textbf{S} contains not only the parameters that describe rotation, but also shearing and scaling.
\par
\textit{In vivo} motion can be quite complex, such that time-dependent deformation fields would be needed for its full characterisation. In practice however, this complete description is often not necessary, and a simplified model is sufficient. For example, respiratory movement is often modelled as periodic motion in one dimension. Rigid body transformations are used to model head motion in neural imaging \parencite{Godenschweger2016}, and affine (as well as rigid body) models have been used to describe motion in cardiac and abdominal imaging \parencite{Nehrke2005,Manke2002,Pipe1999}. While affine models of \textit{in vivo} motion are quite useful, it is important to keep in mind that they are only a useful local approximation for a region of interest, and generally not applicable to the entire field of view.
\par
Nevertheless, in the following investigation we will assume that the points or voxels $\tilde{\textbf{r}}$ in the imaging volume are undergoing motion that can be modelled as an affine transformation. The movement of $\tilde{\textbf{r}}$ to a new position $\tilde{\textbf{r}}^{'}$ can therefore be described by Eq. (\ref{eq1}).

\subsubsection*{Motion and K-space Inconsistencies}

Acquired signal in MRI can be written as
\begin{equation}
	\label{eq2}
	s(\tilde{\textbf{k}}) = \int d\textbf{V} \rho\left(\tilde{\textbf{r}}\right)\operatorname{e}^{-i2\pi\tilde{\textbf{r}}^{T}\tilde{\textbf{k}}}
\end{equation}
where $\tilde{\textbf{r}}$ and $\tilde{\textbf{k}}$ are homogeneous coordinate vectors. The vector $\tilde{\textbf{k}} = \begin{bmatrix}k_{x},&k_{y},&k_{z},&\varphi\end{bmatrix}^{T}$, with $\varphi$ representing an initial phase.
\par
Let $\tilde{\textbf{r}}_0$ and $\tilde{\textbf{k}}_0$ represent the augmented, position and k-space vectors when no motion is present. Now assume that some movement occurs, such that $\tilde{\textbf{r}}_0$ is transformed to $\tilde{\textbf{r}}$ by the affine matrix \textbf{A}, hence $\tilde{\textbf{r}} = \textbf{A}\tilde{\textbf{r}}_0$. The phase factor $\Phi$ is therefore
\begin{equation} \label{eq3}
	\begin{split}
		\Phi & = \tilde{\textbf{r}}^T\tilde{\textbf{k}}_0 \\ 
			 & = (\textbf{A}\tilde{\textbf{r}}_0)^T\tilde{\textbf{k}}_0
		     = {\tilde{\textbf{r}}_0}^T\textbf{A}^T\tilde{\textbf{k}}_0 \\
		     & = {\tilde{\textbf{r}}_0}^T\tilde{\textbf{k}}.
	\end{split}
\end{equation}
An interpretation of the above operations is that we are keeping the body reference frame fixed, and rather than sampling the points $\tilde{\textbf{k}}_0$, we are now sampling the coordinates $\tilde{\textbf{k}}$ in a transformed k-space due to motion (Fig 1.). In the case where \textbf{A} is a rigid body transformation,
\begin{equation} \label{eq4}
	\begin{split}
		\tilde{\textbf{k}} & = \textbf{A}^T\tilde{\textbf{k}}_0
		= \begin{bmatrix}\textbf{R}^T & \textbf{0} \\ \textbf{d}^T & 1\end{bmatrix}\begin{bmatrix}k_{0x}\\k_{0y}\\k_{0z}\\\varphi_0\end{bmatrix}\\
		& = \begin{bmatrix}k_{x}\\k_{y}\\k_{z}\\\Delta_{x}k_{0x} + \Delta_{y}k_{0y} + \Delta_{z}k_{0z} + \varphi_0\end{bmatrix}.
	\end{split}
\end{equation}
From Eq. (\ref{eq4}) we can see that rigid body motion results in a rotation of the original k-space coordinates by $\textbf{R}^T$, and adds a phase term ($\Delta_{x}k_{0x} + \Delta_{y}k_{0y} + \Delta_{z}k_{0z}$) to the initial value $\varphi$. Generality under an affine transformation is not lost here, $\textbf{R}^T$ would simply be replaced by $\textbf{S}^T$. The use of homogeneous coordinates therefore allows for the succinct representation of acquired signal when a motion \textbf{A} occurs:
\begin{equation} \label{eq5}
s(\tilde{\textbf{k}}) = \int d\textbf{V} \rho\left(\tilde{\textbf{r}}_0\right)\operatorname{e}^{-i2\pi{\tilde{\textbf{r}}_0}^{T}\textbf{A}^T\tilde{\textbf{k}}_0}.
\end{equation}
\par
K-space acquisition is sequential, rather than instantaneous, and thus the k-space coordinates have an implicit time dependence. The motion is also a function of time, \textbf{A(t)}. Depending on the acquisition trajectory and \textbf{A(t)}, different portions of k-space will be rotated and phase-shifted relative to other parts. For Cartesian k-space trajectories, these inconsistencies occur across lines in the phase encoding direction. Generally speaking, if these displacements are smooth (such as in the case of slow, continuous motion), blurring will be present in the reconstructed image. Periodic displacements and large rotations can cause periodic phase modulations and large gaps in the k-space data, which results in ghosting.

\subsubsection*{Signal Corruption}
Motion between excitations can have effects other than blurring and ghosting.


\subsection{Intra-view Effects}


\end{document}