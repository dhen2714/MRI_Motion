\documentclass[class=article, crop=false]{standalone}

\begin{document}

The type of motion artifact and its severity depends on the magnitude and type of motion, as well as its temporal relation to the data acquisition method. \textit{In vivo} motion can be broadly categorised into three groups: rigid body motion, elastic motion, or flow. The motion of the head during a brain scan can be described as rigid body motion, and requires six parameters (three for translations and three for rotations) to be fully specified. Elastic motion requires twelve degrees of freedom. It involves compression, stretching and skew as well as rigid body motion, and is observed in abdominal scans. The description of flow can vary, from single parameter models for laminar flow in some blood vessels, to the two or three dimensional vector fields needed to describe blood flow near the heart. 


\subsection{Blurring and Ghosting}

\subsection{Signal and Contrast Corruption}


\end{document}