\documentclass[class=article, crop=false]{standalone}

\begin{document}
	
\subsection{Motion Estimation}
In order to correct for motion, it must first be accurately characterised. Restricting ourselves mainly to head movements, the focus will be on rigid body motion. Motion information can be obtained by the MR scanner itself, through `navigator-based' estimation. Alternatively, external hardware can be used to find motion parameters. While navigator-based estimation has the obvious advantage of not needing additional hardware to the MR scanner, it places restrictions on the pulse sequence which may rule out its use in some situations. These considerations will be discussed in further detail below.

\subsubsection{Navigator-based Estimation}
Navigator based estimation can be further broken down into two categories: navigator echoes and self-navigators. Navigator echoes are additional pulse sequences that are inserted into the host sequence. Self-navigators on the other hand, are specially designed k-space trajectories that oversample portions of k-space in order to extract motion information.
\paragraph{Navigator echoes:}
The navigator echo (NAV) was first introduced by Ehman and Felmlee \parencite*{Ehman1989}. In this work, an additional $\pi$ pulse was inserted into a spin echo sequence, with a corresponding readout in the frequency encoding dimension. The NAV readout was not phase encoded, but could be Fourier transformed and compared with a reference NAV image to determine if bulk motion in the frequency encoding dimension had occurred. The method presented here could only measure translations in one dimension, and assumed the whole FOV moved in unison, but nevertheless was an important step in motion correction.
\par
Since the work of Ehman and Felmlee, more complex navigator echoes have been developed, in order to estimate higher DOF motion. The orbital navigator (oNav), introduced by Fu et al. \parencite*{ZhuoWuFu1995}, presented a way of estimating in plane translations and rotations. The inserted navigator echo sampled the circumference of a circle in k-space. The rotation could be estimated by comparing the magnitude of the raw oNav data with a reference view, and finding the rotation that minimised the difference. Translations could then be determined by looking at the raw phase data. The use of three orthogonal oNavs was later used to estimate motion in six DOF \parencite{Ward2000}, and a combination of two orthogonal NAV echoes with oNav has been shown to be effective in correcting for both in plane and through plane motion in 2D multi-slice imaging \parencite{Lin2014}. The use of oNavs has been limited however, due to their inability to adequately account for motion orthogonal to the navigator sampling trajectory. While this problem was acknowledged by Ward et al., their proposed solution of acquiring multiple oNav sets to iteratively correct for this motion is perhaps too expensive time-wise.
\par
Spherical navigators (sNav) were suggested as an alternative to oNavs that provided similar six DOF motion estimation, but without the same sensitivity to through plane motion effects \parencite{Welch2001,Petrie2005}. Despite being effective in this regard and rapid in acquisition, the estimation of 6 DOF motion parameters (both iteratively and non-iteratively) remains computationally intensive \parencite{Welch2004,Costa2010,Costa2005}. This has restricted the use of sNavs to retrospective motion correction. Recent efforts to reduce computation time have led to the use of template matching as opposed to registration of spheres \parencite{Liu2011,Johnson2016}. Here, extra time is needed before the scan to acquire baseline rotations templates that cover the anticipated range of rotation. While the technique in \cite{Johnson2016} was implemented for retrospective correction, the authors suggested that a total sNav processing time of 35 ms per shot could be achieved with optimisation, making it feasible for prospective motion correction.
\par
Another solution to the problem of through-plane motion is the use of a cloverleaf navigator \parencite{VanDerKouwe2006}. Before the scan, a map of features around the cloverleaf navigator is acquired, which, like the template sNav methods mentioned previously, requires a few seconds in which the patient is motionless. This map, which is a series of slightly rotated cloverleaf trajectories, can then be used to correct out-of-plane errors. The advantage that the cloverleaf navigator has over sNavs is its time requirement of 4.2 ms. It can be inserted before or after gradient encoding, without the need for an extra rf pulse. The cloverleaf trajectory samples k-space far less densely than sNavs however, so acquired motion estimates are more susceptible to noise.
\par
A common thread that is shared amongst oNavs, sNavs and cloverleaf trajectories is that they all estimate motion with the use of the raw k-space data. This has some advantages. Rotations rotate k-space magnitude and phase, while translations affect phase only, and thus motion estimation in k-space is a decoupled problem. Also, k-space does not have to be sampled fully to estimate motion, saving time in acquisition. A potentially serious downside though is the presence of non-rigid motion. Parts of the head such as the jaw or neck may move relative to the brain, adversely affecting k-space based rigid body motion estimates.
\par
A technique that is robust to the previously mentioned problem is PROMO (PROspective MOtion correction) \parencite{White2010a}. It uses image space rather than k-space to estimate motion. Three orthogonal spiral navigators are used to reconstruct three  images. A region of interest is drawn around the brain in a reference navigator image set, and voxels outside this region are ignored in tracking. The motion of the brain is then tracked in image space using an Extended Kalman Filter (EKF) approach. Non-rigid motion of regions outside the brain therefore does not affect motion estimates. Acquisition of the spiral navigators and their reconstruction requires about 48 ms of time within the pulse sequence.
\par
A natural extension to the idea of acquiring three orthogonal images is acquiring entire volumes, and using image registration for motion estimation. Volumetric EPI navigators (vNavs) were first used in single voxel spectroscopy \parencite{Hess2011}, and later for neuroanatomical imaging \parencite{Tisdall2012}, as well as also diffusion tensor imaging \parencite{MRM:MRM23314}. The scan times of the navigators, inserted once per TR, were 275 ms for \cite{Tisdall2012} and 406 ms for \cite{MRM:MRM23314}. The significant time overhead required for the acquisition of vNAVs therefore rules them out for insertion into host sequences such as FLASH and EPI, unless the navigators are interleaved between TRs.
\par 
In addition to k-space navigators and image-space navigators, there has also been interest in non-encoded free induction decay (FID) navigators \parencite{Waszak2016,MRM:MRM22797}. FID navigators are ultrashort ($<$100 $\mu$s) pulses that are not gradient encoded, meaning they can fit quite comfortably into most sequences. Because they are not encoded, the signals themselves contain no spatial information, and so are not currently used to measure motion, but rather to simply detect it. Motion causes changes in the FID signal received by each RF coil. This can then be compared to a `motion-free' reference to detect motion, in which case a separate navigator (like the ones discussed previously) is inserted into the sequence. The use of FID navigators can thus decrease scan times by facilitating selective use of navigator echoes. 


\paragraph{Self-navigation:}

\subsubsection{External Hardware-based Estimation}

\subsection{Retrospective Motion Correction}

\subsection{Prospective Motion Correction}

\end{document}