\documentclass[class=article, crop=false]{standalone}

\begin{document}
Dealing with subject motion is an important and ongoing challenge in MRI. For many imaging sequences, the acquisition of lines in k-space, or entire imaging volumes, is under the assumption of stationary spin density. Given a live subject, and scan times that are usually in the order of minutes, this assumption is often unreasonable. Motion during a scan can occur in many different ways, depending on the anatomy being imaged. Examples of such occurrences include: movement of the head during brain MRI, cardiac motion, respiration, and flowing of blood. The large amount of noise produced by the scanner coils and the restrictive bore size also means that patient comfort is often not ideal. This can exacerbate movement, especially amongst paediatric and elderly patients. The effect that motion can have on the final image can be severe, and has long been identified as a problem \parencite{Bellon1986}.
\par
Motion affected images can exhibit a variety of artifacts, such as blurring, ghosting, geometric distortions, reduced signal to noise ratio, and aberrant fluctuations in intensity. Rather than being simply cosmetic, these artifacts can have detrimental effects on any measurements derived from the image \parencite{Gedamu2012,LeBihan2006,Reuter2015a}, as well as its overall diagnostic quality \parencite{Dantendorfer1997}. The presence of artifacts often means repeated scans, leading to increases in hospital and personnel cost, and can adversely affect patient comfort. In a study involving 192 clinical MRI examinations, Andre et al. \parencite*{Andre2015} found that roughly 20\% involved repeat scans due to motion degradation. They estimated that the institutional cost due to patient motion was in the order of 100,000 USD per scanner, per year. An effective strategy for dealing with motion could thus provide better patient outcomes, and reduce clinical costs associated with machine operation and personnel.
\par
While many techniques have been developed over the years to address the problem of motion, the ever-growing list of data acquisition methods in MRI and their varying interactions with different types of motion has made it difficult to find a ‘one-size-fits-all’ solution. As described by Zaitsev, Maclaren and Herbst \parencite*{Zaitsev2015a}, there exists a ‘toolbox’ of methods, each with varying, and sometimes overlapping, domains of applicability. Before delving into this toolbox, we will first try to build an understanding as to how different types of motion can affect measured signal, and how these effects can lead to artifacts in the reconstructed image.

\end{document}